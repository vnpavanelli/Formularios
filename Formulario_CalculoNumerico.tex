\documentclass{article}
\usepackage[utf8]{inputenc}
\usepackage[brazil]{babel}
\usepackage{graphicx}
\usepackage{caption}
\usepackage{amsmath}
\usepackage[margin=0.5in]{geometry}
\usepackage{mathrsfs}
\usepackage{amssymb}
\usepackage{textcomp}
\usepackage{booktabs,tabularx}

\begin{document}
\begin{center}
Vinicius Pavanelli Vianna - NUSP 3155408 \hfill\\
\end{center}
\begin{minipage}[!t]{.5\textwidth}
%\raggedright
\begin{tabularx}{\textwidth}{lcccc}
\toprule
\multicolumn{5}{c}{IEEE 754 (bits)}\\
Tamanho & Sinal & expoente & polarização & mantissa\\
\midrule
32 & 1 & 8 & 127 & 32\\
64 & 1 & 11 & 1023 & 52\\
\midrule
Formato & \multicolumn{2}{c}{Single} & \multicolumn{2}{c}{Double}\\
\midrule
$E_{min}$ & \multicolumn{2}{c}{$-126$} & \multicolumn{2}{c}{$-1022$}\\
$E_{max}$ & \multicolumn{2}{c}{$127$} & \multicolumn{2}{c}{$1023$} \\
menor \textnumero & \multicolumn{2}{c}{$\approx 1,2 \times 10^{-38}$} & \multicolumn{2}{c}{$\approx 2,2 \times 10^{-308}$} \\ 
maior \textnumero & \multicolumn{2}{c}{$\approx 3,4 \times 10^{38}$} & \multicolumn{2}{c}{$\approx 1,8 \times 10^{308}$} \\
$\epsilon$ & \multicolumn{2}{c}{$2^{-23} \approx 1,2 \times 10^{-7}$} & \multicolumn{2}{c}{$2^{-52} \approx 2,2 \times 10^{-16}$} \\
Precisão & \multicolumn{2}{c}{$\approx$ 7 digitos decimais} & \multicolumn{2}{c}{$\approx$ 16 digitos decimais} \\
\midrule
\multicolumn{5}{c}{$ F(\beta, t, m, M) \implies x = s \times \beta^e \qquad e \in [-m, M]$}\\
\multicolumn{5}{c}{$\beta^{-1}(1-\frac{1}{2}\beta{-t}) \leq s \leq 1 - \frac{1}{2}\beta{-t}$}\\
\bottomrule
\end{tabularx}
\\
\begin{tabularx}{\textwidth}{c}
\toprule
MMQ\\
\midrule
$f(x) \approx g(x)$
$r(x_i) = \sum_{i=1}^{n} [f(x_i) - g(x_i) ]^2$ \\
$r(x) = \int_c^d [f(x)-g(x)]^2 dx$\\
\hline
$\begin{bmatrix}
\langle g_1,g_1 \rangle & \langle g_1,g_2 \rangle & \hdots & \langle g_1,g_m \rangle \\
\langle g_2,g_1 \rangle & \langle g_2,g_2 \rangle & \hdots & \langle g_2,g_m \rangle \\
\vdots & \vdots & \ddots & \vdots \\
\langle g_m,g_1 \rangle & \langle g_m,g_2 \rangle & \hdots & \langle g_m,g_m \rangle \\
\end{bmatrix}$
$\begin{bmatrix}
a_1 \\
a_2 \\
\vdots \\
a_m \\
\end{bmatrix}$
= $\begin{bmatrix}
\langle g_1,f \rangle \\
\langle g_2,f \rangle \\
\vdots \\
\langle g_m,f \rangle \\
\end{bmatrix}$\\
\midrule
Linearizações\\
\midrule
$g(x) = ae^{bx} \implies \ln(g(x)) = \ln(a) + bx$\\
$g(x) = a10^{bx} \implies \log(g(x))= \log(a) + bx$\\
$g(x) = ax^b \implies \ln(g(x)) = \ln(a) + b\ln(x)$\\
$g(x) = \frac{1}{a+bx} \implies \frac{1}{g(x)} = a + bx$\\
$g(x) = \frac{ax}{b+x} \implies \frac{1}{g(x)} = \frac{b}{a}\frac{1}{x} + \frac{1}{a}$\\
\midrule
Resíduos MMQ\\
\hline
$r(x_i) = \displaystyle\sum_{i=1}^{n} [f(x_i) - g(x_i) ]^2$ \\
$R^2 = 1 - \frac{SQ_{res}}{SQ_{tot}}$\\
$\bar{y} = \frac{1}{n}\displaystyle\sum_{i=1}^{n}f(x_i)$ (ponto médio) \\
$SQ_{tot} = \displaystyle\sum_{i=1}^{n} [g(x_i) - \bar{y}]^2$ (erro total) \\
$SQ_{res} = \displaystyle\sum_{i=1}^{n} [f(x_i) - \bar{y}]^2$ (erro residual) \\
\midrule
Serie Fourier\\
\midrule
$
\begin{aligned}
s_N(x) &= \frac{a_0}{2} + \displaystyle \sum_{n=1}^{N} \left( a_n \cos ( \frac{2\pi n x}{P} ) + b_n\sin ( \frac{2\pi n x}{P}) \right)\\
&= \displaystyle \sum_{n=-N}^{N} c_n e^{i\frac{2\pi n x}{P}}\\
\end{aligned}
$\\
$ a_n = \frac{2}{P} \displaystyle \int_{x_0}^{x_0+P} s(x) \cos(\frac{2\pi n x}{P}) dx $\\
$ b_n = \frac{2}{P} \displaystyle \int_{x_0}^{x_0+P} s(x) \sin(\frac{2\pi n x}{P}) dx $\\
$ c_n = \frac{1}{P} \displaystyle \int_{x_0}^{x_0+P} s(x) e^{-i\frac{2\pi n x}{P}} dx $\\
\bottomrule
\end{tabularx}\\
\end{minipage}
\hfill
\noindent
\begin{minipage}[!t]{.5\textwidth}
\begin{tabularx}{\textwidth}{c}
\toprule
Forma interpoladora de Lagrange\\
\midrule
$f(x) \approx P_1(x) = a_0(x-x_1) + a_1(x-x_0)$\\
$a_0 = \frac{f(x_0)}{x_0-x_1} ; a_1 = \frac{f(x_1)}{x_1-x_0}$ \\
%\begin{equation}
$\begin{aligned}
f(x) \approx P_2(x) &= (x-x_1)(x-x_2)\frac{f(x_0)}{(x_0-x_1)(x_0-x_2)}\\ 
&+ (x-x_0)(x-x_2)\frac{f(x_1)}{(x_1-x_0)(x_1-x_2)} \\
&+ (x-x_0)(x-x_1)\frac{f(x_2)}{(x_2-x_0)(x_2-x_1)}\\
\end{aligned}$\\
%\end{equation}
\hline
$ L_i(x) = \displaystyle\prod_{k=0,k \neq i}^{n} \frac{x-x_k}{x_i-x_k} , i = 0 \dots n$\\
$f(x) \approx P_n(x)= \displaystyle\sum_{k=0}^n f(x_k)L_k(x)$\\
\midrule
Forma interpoladora de Newton\\
\midrule
$\begin{aligned}
f(x) \approx P_n(x) &= a_1 \\
&+ a_2(x-x_0) + a_3(x-x_0)(x-x_1) + \dots\\
&+ a_n(x-x_0)(x-x_1)\dots(x-x_{x-1})
\end{aligned}$\\
$a_1 = f(x_0)$\\
$a_2 = \frac{f(x_1)-f(x_0)}{x_1-x_0}$\\
$a_3 = \frac{ \frac{f(x_2)-f(x_1)}{x_2-x_1} - \frac{f(x_1)-f(x_0)}{x_1-x_0}   }{x_2-x_0}$\\
\midrule
Forma Interpoladora de Newton - Diferenças divididas\\
\midrule
$f[x_i] = f(x_i)$\\
$f[x_i,x_{i+1}] = \frac{f[x_{i+1}] - f[x_i]}{x_{i+1}-x_i}$\\

$f[x_i,x_{i+1},x_{i+2}] = \frac{f[x_{i+1},x_{i+2}] - f[x_i,x_{i+1}]}{x_{i+2} - x_i}$\\


$f[x_i,x_{i+1}\dotsc x_{i+k}] = \frac{ f[x_{i+1}\dotsc x_{i+k}] - f[ x_i, x_{i+k-1} ]   }{x_{i+k} - x_i}$\\
$\begin{aligned}
f(x) \approx P_n(x) &= f[x_0] + f[x_0,x_1](x-x_0) \\
&+ f[x_0,x_1,x_2](x-x_0)(x-x_1) + \dots\\
&+ f[x_0,x_1 \dotsc x_n](x-x_0)(x-x_1)\dots(x-x_{n-1})
\end{aligned}$\\


\midrule
Erro na interpolação\\
\midrule
$f(x) = P_n(x) + R_n(x)$\\
$ R_n(x) = \frac{ (x-x_0)(x-x_1)\dots (x-x_n)}{(n+1)!}f^{(n+1)}(\xi(x)), x_0 < \xi < x_n$\\
$ M > 0 $ e $ \displaystyle \max_{t\in [x_0,x_n]} |f^{(n+1)}(t)| < M$\\
$ |R_n(x)| \leq \frac {|x-x_0||x-x_1|\dots |x-x_n|}{(n+1)!}M $\\
$ |R_n(x)| \leq \frac {|x-x_0||x-x_1|\dots |x-x_n|}{(n+1)!}f[x_0,x_1 \dotsc x_n,x_{n+1}] \iff \exists x_{n+1} $ \\
\bottomrule
\end{tabularx}\\
\end{minipage}
\newpage






\vspace{1em}
\begin{tabularx}{\textwidth}{lc}
\toprule
\multicolumn{2}{c}{Métodos de Integração Numérica}\\
\midrule
Ponto Central & $ \displaystyle \int_a^b f(x) dx \approx h \displaystyle \sum_{i+1}{N} f \left(\frac{x_i + x_{i+1}}{2} \right)$ \\
Trapézio & $ \displaystyle \int_a^b f(x)dx \approx \frac{h}{2} \left[ f(x_0) + 2\left( f(x_1) + \dots + f(x_{N-1}) \right) + f(x_N) \right] $\\

Simpson 1/3 & $ \displaystyle \int_a^b f(x)dx \approx \frac{h}{3} \left[ f(x_0) + 4f(x_1) + 2f(x_2) + 4f(x_3) + \dots + 2f(x_{2N-2}) + 4f(x_{2N-1}) + f(x_{2N}) \right] $\\

Simpson 3/8 & $ 
\begin{aligned}
\displaystyle \int_a^b f(x)dx \approx &\frac{3h}{8} [ f(x_0) + 3f(x_1) + 3f(x_2) + 2f(x_3) + \dots\\
&+ 2f(x_{3N-3}) + 3f(x_{3N-2}) + 3f(x_{3N-1}) + f(x_{3N}) ] 
\end{aligned}
$\\
\midrule
\multicolumn{2}{c}{Quadratura de Gauss}\\
\midrule
Geral & $ \displaystyle \int_a^b w(x)f(x)dx \approx \displaystyle \sum_{i=1}{n} C_if(x_i) $ ($C_i$ são os pesos e $x_i$ são os pontos tabelados \\
Gauss-Legendre & $w(x)=1, a=-1, b=1$, $n=2 \implies C_1 = C_2 = 1, x_1 =-0,57735027 e x_2 = -0,57735027$\\
& $ n=2 \implies \displaystyle \int_{-1}^1 f(x)dx \approx f(-0,57735027) + f(0,57735027) $\\
Gauss-Chebyshev & $w(x) = \frac{1}{\sqrt{1-x^2}}, a=-1, b=1$ \\
Gauss-Laguerre & $ w(x) = e^{-x}, a=0, b=\infty$\\
Gauss-Hermite & $ w(x) = e^{-x^2}, a=-\infty, b=\infty$\\
\midrule
\multicolumn{2}{c}{Erro na Integração Numérica}\\
\midrule
Trapézio & $R_1(f) = -\frac{(b-a)}{12}h^2f''(\xi), \xi \in (x_0,x_1) $\\
Simpson 1/3 & $R_2(f) = -\frac{(b-a)}{180}h^4f^{(4)}(\xi), \xi \in (x_0,x_{2N})$\\
Simpson 3/8 & $R_3(f) = -\frac{(b-a)}{80}h^4f^{(4)}(\xi), \xi \in (x_0,x_{3N})$\\
\bottomrule
\end{tabularx}\\

\vspace{2em}
\begin{tabularx}{.5\textwidth}{c}
\toprule
Lista de Exercícios\\
\midrule
$ y_n = \int_0^1 \frac{x^n}{x+a} dx \implies \int_0^1 \frac{x^1}{1+a}dx < y_n < \int_0^1 \frac{x^n}{a}dx$\\
$ \frac{1}{(n+1)(1+a)} < y_n < \frac{1}{(n+1)a}$\\
$ x^n = [(x+a)-a)]^n = \displaystyle\sum_{k=0}^{n} (-1)^k \left( \begin{array}{c}
n\\
k
\end{array}
\right) (x+a)^{n-k}a^k$\\

$
\begin{aligned}
 y_n &= \displaystyle \int_0^1 \displaystyle \sum_{k=0}^{n} (-1)^k \left( \begin{array}{c}
n\\
k
\end{array}
\right) (x+a)^{n-k-1}a^k dx\\

&= \displaystyle \sum_{k=0}^{n} (-1)^k a^k \left( \begin{array}{c}
n\\
k
\end{array}
\right) \displaystyle \int_0^1 (x+a)^{n-k-1} dx\\

&= \displaystyle \sum_{k=0}^{n-1} (-1)^k a^k \left( \begin{array}{c}
n\\
k
\end{array}
\right) \displaystyle \int_0^1 (x+a)^{n-k-1} dx\\

&+ (-1)^n a^n \left( \begin{array}{c}
n\\
n
\end{array}
\right) \displaystyle \int_0^1 (x+a)^{-1}dx\\
\end{aligned}
$\\
\midrule
\end{tabularx}
\begin{tabularx}{.5\textwidth}{c}
\toprule
Livro Neide\\
\midrule
$ I_n = e^{-1} \displaystyle \int_0^1 x^n e^x dx \implies I_n < e^{-1} \max_{0\leq x \leq 1}(e^x) \displaystyle \int_0^1 x^n dx < \frac{1}{n+1} $\\
$ I_n = e^{-1} \left\{ \displaystyle \left[x^n e^x \right]_0^1 - \displaystyle \int_0^1 n x^{n-1} e^x dx \right\} $\\
$ c_r = \frac{|P'(x)}{P(x)}\qquad c_r \leq 1 \implies$ bem condicionado \\
\midrule
Formulas Gerais\\
\midrule
$ 
\left( \begin{array}{c}
n\\
k
\end{array}
\right) = \frac{n!}{k!(n-k)!}$\\
$ f(x) \approx \displaystyle \sum_{n=0}^{\infty} \frac{f^{(n)}(a)}{n!}(x-a)^n $\\
$ e^{x} = \sum^{\infty}_{n=0} \frac{x^n}{n!} = 1 + x + \frac{x^2}{2!} + \frac{x^3}{3!} + \cdots $\\
$ \displaystyle \sum_{k=0}^{n-1} ar^k= a \left(\frac{1-r^{n}}{1-r}\right) $\\
\bottomrule
\end{tabularx}

























\end{document}