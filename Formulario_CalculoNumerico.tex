\documentclass{article}
\usepackage[utf8]{inputenc}
\usepackage[brazil]{babel}
\usepackage{graphicx}
\usepackage{caption}
\usepackage{amsmath}
\usepackage[margin=0.5in]{geometry}
\usepackage{mathrsfs}
\usepackage{amssymb}
\usepackage{textcomp}

\begin{document}
Vinicius Pavanelli Vianna - NUSP 3155408
\vspace{1em}\\
\\
\begin{tabular}{lcccc}
\multicolumn{5}{c}{IEEE 754 (bits)}\\
Tamanho & Sinal & expoente & polarização & mantissa\\
\hline
32 & 1 & 8 & 127 & 32\\
64 & 1 & 11 & 1023 & 52\\
\end{tabular}
%\vspace{1em}
\begin{tabular}{ccc}
Formato & Single & Double\\
\hline
$E_{min}$ & $-126$ & $-1022$\\
$E_{max}$ & $127$ & $1023$ \\
menor \textnumero & $\approx 1,2 \times 10^{-38}$ & $\approx 2,2 \times 10^{-308}$ \\ 
maior \textnumero & $\approx 3,4 \times 10^{38}$ & $\approx 1,8 \times 10^{308}$ \\
$\epsilon$ & $2^{-23} \approx 1,2 \times 10^{-7}$ & $2^{-52} \approx 2,2 \times 10^{-16}$ \\
Precisão & $\approx 7 digitos decimais$ & $\approx 16 digitos decimais$ \\
\end{tabular}

\vspace{1em}

\begin{tabular}{c}
MMQ\\
\hline
$f(x) \approx g(x)$
$r(x_i) = \sum_{i=1}^{n} [f(x_i) - g(x_i) ]^2$ \\
$r(x) = \int_c^d [f(x)-g(x)]^2 dx$\\
\hline
$\begin{bmatrix}
\langle g_1,g_1 \rangle & \langle g_1,g_2 \rangle & \hdots & \langle g_1,g_m \rangle \\
\langle g_2,g_1 \rangle & \langle g_2,g_2 \rangle & \hdots & \langle g_2,g_m \rangle \\
\vdots & \vdots & \ddots & \vdots \\
\langle g_m,g_1 \rangle & \langle g_m,g_2 \rangle & \hdots & \langle g_m,g_m \rangle \\
\end{bmatrix}$
$\begin{bmatrix}
a_1 \\
a_2 \\
\vdots \\
a_m \\
\end{bmatrix}$
= $\begin{bmatrix}
\langle g_1,f \rangle \\
\langle g_2,f \rangle \\
\vdots \\
\langle g_m,f \rangle \\
\end{bmatrix}$\\
\hline
\end{tabular}
%\vspace{1em}
\begin{tabular}{c}
Linearizações\\
\hline
$g(x) = ae^{bx} \implies \ln(g(x)) = \ln(a) + bx$\\
$g(x) = a10^{bx} \implies \log(g(x))= \log(a) + bx$\\
$g(x) = ax^b \implies \ln(g(x)) = \ln(a) + b\ln(x)$\\
$g(x) = \frac{1}{a+bx} \implies \frac{1}{g(x)} = a + bx$\\
$g(x) = \frac{ax}{b+x} \implies \frac{1}{g(x)} = \frac{b}{a}\frac{1}{x} + \frac{1}{a}$\\
\end{tabular}\\
%\vspace{1em}
\begin{tabular}{c}
Resíduos MMQ\\
\hline
$r(x_i) = \displaystyle\sum_{i=1}^{n} [f(x_i) - g(x_i) ]^2$ \\
$R^2 = 1 - \frac{SQ_{res}}{SQ_{tot}}$\\
$\bar{y} = \frac{1}{n}\displaystyle\sum_{i=1}^{n}f(x_i)$ (ponto médio) \\
$SQ_{tot} = \displaystyle\sum_{i=1}^{n} [g(x_i) - \bar{y}]^2$ (erro total) \\
$SQ_{res} = \displaystyle\sum_{i=1}^{n} [f(x_i) - \bar{y}]^2$ (erro residual) \\
\end{tabular}	
%\vspace{1em}
\begin{tabular}{c}
Forma interpoladora de Lagrange\\
\hline
$f(x) \approx P_1(x) = a_0(x-x_1) + a_1(x-x_0), a_0 = \frac{f(x_0)}{x_0-x_1} ; a_1 = \frac{f(x_1)}{x_1-x_0}$ \\
%\begin{equation}
$\begin{aligned}
f(x) \approx P_2(x) &= (x-x_1)(x-x_2)\frac{f(x_0)}{(x_0-x_1)(x_0-x_2)}\\ 
&+ (x-x_0)(x-x_2)\frac{f(x_1)}{(x_1-x_0)(x_1-x_2)} \\
&+ (x-x_0)(x-x_1)\frac{f(x_2)}{(x_2-x_0)(x_2-x_1)}\\
\end{aligned}$\\
%\end{equation}
\hline
$ L_i(x) = \displaystyle\prod_{k=0,k \neq i}^{n} \frac{x-x_k}{x_i-x_k} , i = 0 \dots n$\\
$f(x) \approx P_n(x)= \displaystyle\sum_{k=0}^n f(x_k)L_k(x)$\\
\end{tabular}\\

\vspace{1em}

\begin{tabular}{l}
Forma interpoladora de Newton\\
\hline
$\begin{aligned}
f(x) \approx P_n(x) &= a_1 \\
&+ a_2(x-x_0) + a_3(x-x_0)(x-x_1) + \dots\\
 &+ a_n(x-x_0)(x-x_1)\dots(x-x_{x-1})
\end{aligned}$\\
$a_1 = f(x_0)$\\
$a_2 = \frac{f(x_1)-f(x_0)}{x_1-x_0}$\\
$a_3 = \frac{ \frac{f(x_2)-f(x_1)}{x_2-x_1} - \frac{f(x_1)-f(x_0)}{x_1-x_0}   }{x_2-x_0}$\\
\end{tabular}
%\vspace{1em}
\begin{tabular}{c}
Forma Interpoladora de Newton - Diferenças divididas\\
\hline
$f[x_i] = f(x_i)$\\
$f[x_i,x_{i+1}] = \frac{f[x_{i+1}] - f[x_i]}{x_{i+1}-x_i}$\\

$f[x_i,x_{i+1},x_{i+2}] = \frac{f[x_{i+1},x_{i+2}] - f[x_i,x_{i+1}]}{x_{i+2} - x_i}$\\


$f[x_i,x_{i+1}\dotsc x_{i+k}] = \frac{ f[x_{i+1}\dotsc x_{i+k}] - f[ x_i, x_{i+k-1} ]   }{x_{i+k} - x_i}$\\
$\begin{aligned}
f(x) \approx P_n(x) &= f[x_0] + f[x_0,x_1](x-x_0) \\
&+ f[x_0,x_1,x_2](x-x_0)(x-x_1) + \dots\\
&+ f[x_0,x_1 \dotsc x_n](x-x_0)(x-x_1)\dots(x-x_{n-1})
\end{aligned}$\\


\end{tabular}
\vspace{1em}\\
\begin{tabular}{c}
Erro na interpolação\\
\hline
$f(x) = P_n(x) + R_n(x)$\\
$ R_n(x) = \frac{ (x-x_0)(x-x_1)\dots (x-x_n)}{(n+1)!}f^{(n+1)}(\xi(x)), x_0 < \xi < x_n$\\
$ M > 0 $ e $ \displaystyle \max_{t\in [x_0,x_n]} |f^{(n+1)}(t)| < M$\\
$ |R_n(x)| \leq \frac {|x-x_0||x-x_1|\dots |x-x_n|}{(n+1)!}M $\\
$ |R_n(x)| \leq \frac {|x-x_0||x-x_1|\dots |x-x_n|}{(n+1)!}f[x_0,x_1 \dotsc x_n,x_{n+1}] \iff \exists x_{n+1} $ \\
\end{tabular}\\
\vspace{1em}


\begin{tabular}{lc}
\multicolumn{2}{c}{Métodos de Integração Numérica}\\
\hline
Ponto Central & $ \displaystyle \int_a^b f(x) dx \approx h \displaystyle \sum_{i+1}{N} f \left(\frac{x_i + x_{i+1}}{2} \right)$ \\
Trapézio & $ \displaystyle \int_a^b f(x)dx \approx \frac{h}{2} \left[ f(x_0) + 2\left( f(x_1) + \dots + f(x_{N-1}) \right) + f(x_N) \right] $\\

Simpson 1/3 & $ \displaystyle \int_a^b f(x)dx \approx \frac{h}{3} \left[ f(x_0) + 4f(x_1) + 2f(x_2) + 4f(x_3) + \dots + 2f(x_{2N-2}) + 4f(x_{2N-1}) + f(x_{2N}) \right] $\\

Simpson 3/8 & $ \displaystyle \int_a^b f(x)dx \approx \frac{3h}{8} \left[ f(x_0) + 3f(x_1) + 3f(x_2) + 2f(x_3) + \dots + 2f(x_{3N-3}) + 3f(x_{3N-2}) + 3f(x_{3N-1}) + f(x_{3N}) \right] $\\

\end{tabular}\\

\vspace{1em}

\begin{tabular}{lc}
\multicolumn{2}{c}{Erro na Integração Numérica}\\
\hline
Trapézio & $R_1(f) = -\frac{(b-a)}{12}h^2f''(\xi), \xi \in (x_0,x_1) $\\
Simpson 1/3 & $R_2(f) = -\frac{(b-a)}{180}h^4f^{(4)}(\xi), \xi \in (x_0,x_{2N})$\\
Simpson 3/8 & $R_3(f) = -\frac{(b-a)}{80}h^4f^{(4)}(\xi), \xi \in (x_0,x_{3N})$\\
\end{tabular}\\
\vspace{1em}

\begin{tabular}{c}
Quadratura de Gauss\\
\hline
$ \displaystyle \int_a^b w(x)f(x)dx \approx \displaystyle \sum_{i=1}{n} C_if(x_i) $ ($C_i$ são os pesos e $x_i$ são os pontos tabelados \\
\hline
Gauss-Legendre: $w(x)=1, a=-1, b=1$, $n=2 \implies C_1 = C_2 = 1, x_1 =-0,57735027 e x_2 = -0,57735027$\\
$ n=2 \implies \displaystyle \int_{-1}^1 f(x)dx \approx f(-0,57735027) + f(0,57735027) $\\
Gauss-Chebyshev: $w(x) = \frac{1}{\sqrt{1-x^2}}, a=-1, b=1$ \\
Gauss-Laguerre: $ w(x) = e^{-x}, a=0, b=\infty$\\
Gauss-Hermite: $ w(x) = e^{-x^2}, a=-\infty, b=\infty$\\
\end{tabular}


























\end{document}